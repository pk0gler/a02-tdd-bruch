%!TEX root=../document.tex

\section{Einführung}
Diese Übung soll einen Einblick in die zahlreichen Design Patterns der Softwareentwicklung geben. Sowie die dementsprechend meist verbreitetsten kurz beschreiben und mit Beispielen erläutern.

\subsection{Ziele}
Das Ziel dieser Ausarbeitung ist ein allgemeines Verständnis für die wichtigsten DesignPatterns. Sowhol in der Verwendung als auch im theoretischen Verständnis soll es möglich sein diese umzusetzen.

\subsection{Voraussetzungen}
\begin{itemize}
	\item Grundlagen in Java \& Python
	\item Grundlagen UML
\end{itemize}

\subsection{Aufgabenstellung}
Folgende Punkte müssen in der Ausarbeitung / dem Protokoll enthalten sein.
\begin{itemize}
	\item UML-Klassendiagramm der verwendeten Architektur inkl. Beschreibung
	\item Kurze allgemeine Ausarbeitung zu Design Patterns
	\begin{itemize}
		\item[\Checkmark] Wie können Design Patterns unterteilt werden
		\item[\Checkmark] Wozu Design Patterns
		\item[\Checkmark] Übersicht existierender Design Patterns
	\end{itemize}
	\item Ausarbeitung zum Decorator Pattern
	\begin{itemize}
		\item[\Checkmark] Allgemeines Klassendiagramm
		\item[\Checkmark] Grundzüge des Design Patterns (wichtige Operationen etc.) am Beispiel des implementierten Programms inkl. spezielles Klassendiagramm
		\item[\Checkmark] Vor- und Nachteile
	\end{itemize}
	\item Ausarbeitung zu einem der folgenden Design Patterns: Observer, Abstract Factory, Strategy
	\begin{itemize}
		\item[\Checkmark] siehe oben -> Decorator Pattern
	\end{itemize}
\end{itemize}
\clearpage
